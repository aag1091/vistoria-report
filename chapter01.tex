% !TEX encoding = UTF-8 Unicode
%!TEX root = thesis.tex
% !TEX spellcheck = en-US
%%=========================================
\chapter{Introduction}
The single most important pedagogical task in the teaching of history at either the secondary or post-secondary level is to convey the centrality of what historians call the “historiography” or the record of historical writing on any given theme or problem.  The historiography – of, say, the American Civil War, dietary customs, democracy in Pakistan, industrialization, fashion and dress, the Cuban Revolution – constitutes what historians think they know.  Therefore, to “know” or to understand the history of anything in particular means to know its historiography.

%%=========================================
\section{Interpretations}
Of the many facets of understanding that the historiography provides to the student of history, the single most consequential is the diversity of interpretations that historians have proposed over the years.  An interpretation is the “solution” that a historian proposes to a particular historiographical problem.  In effect, therefore, to learn history is to learn interpretations – and at the same time, to learn how to compare, criticize, evaluate and refine them, while applying them to devise new hypothesis and counter-interpretations.

%%=========================================
\section{Typical Categories of Information}
In building their interpretations, historians invariably deploy other closely-linked kinds of information, typically categorized in the following ways:
%%=========================================
\subsection{Problem}
This is the dilemma or question, thrown up by the historiography, that a given interpretation responds to.  The problem can most conveniently be framed as a question.  It can be very expansive or very narrow in scope.  Examples:  ``What Causes Social Revolution?'' or ``How Can the Victory of the Cuban Revolutionaries in 1959 be Explained''  One problem leads to another, so that we are also likely to ask:  ``How Did the U.S. Government Respond to the Victory of the Cuban Revolutionaries in 1959?''
%%=========================================
\subsection{Periods/Trends}
These refer to historical tendencies that the historiography assigns, often in a rough way, around some range of years, e.g., industrialization, democratization, the Enlightenment, free-trade imperialism, and so on.
%%=========================================
\subsection{Concepts}
These are the abstractions that historians typically apply as they seek to understand a problem, e.g., class, status, democracy, economic growth, development, human rights, absolutism, authoritarianism.  The concepts at stake will depend in part on the theories and interpretations proposed, though the concepts will also have stand-alone analytical utility.
%%=========================================
\subsection{Theories}
The explanatory tools (cited or implied) that historians typically resort to in manipulating concepts and building their interpretations.  Examples include one or more of the many theories, or categories of theories, that posit explanations of the origins of social revolution, e.g., ``structuralism'' or  ``Marxism'' or theories associated with an individual investigator, e.g. Tilly's theory of collective action, Davies' J-Curve or Scott’s moral economy.

\chapter{Existing Systems}
There are not any many system which can be used by historians for performing “historiography”. There exists a few systems which depict some kind of history, but most of them focus on only a single characteristics of the history.

The visualizations can be divided into two categories as follows:-
\section{Custom Built Visualiaztions}
Custom built to highlight particular characteristic of a history. Examples can be found here \url{http://dataremixed.com/2014/02/visualizing-history/}.
\subsection{A PRESIDENTIAL GANTT CHART}
Visualizing Presidential History is plotted from data collected from wikipedia. It only focuses on the founding fathers, and the various wars and movements that occurred over the course of time.
\subsection{Visualizing the History of Civilizations}
This visualization only focuses on Civilizations that existed in past.

\section{Timeline Based Visualizations}
These visualization uses timelines to plot events over a certain periodx.
\subsection{histography.io}
It plots data collected from wikipedia to plot all historical events. The amount of data is overwhelming and is very difficult to customize to fit an Historians needs.
\subsection{Timeline of Historiography and Emperors}
This can be found here \url{http://www.tiki-toki.com/timeline/entry/489025/Timeline-of-Historiography-and-Emperors/}. It shows data for various Roman Emperors on a timeline and links to page showing information for each emperor.

\chapter{Proposed System}
Proposed system should be able to allow users represent a set histographical data into beautiful visualization. All data set should conform typical categories mentioned earlier. The visualizations built using the data set should make historians and students life easier in understanding history and deriving their own interpretations. Lets look at the application stack and proposed pages and their screenshots.
\section{Application Stack}
\begin{itemize}
  \item Database - PostgreSQL
  \item Backend - Ruby On Rails
  \item Frontend - HTML and CSS, Jquery and Zurb Foundation
  \item Heroku - cloud service
\end{itemize}
\section{Proposed pages and application screenshots}
Single page application showing a timline at top and Event data in the middle and other typical categories shown below it. Please refer the figure below.
\\
\includegraphics[width=17cm,height=17cm,keepaspectratio]{fig/vistoria}

A carousel for browsing event in the middle of the page below timeline. Interpretations are shown at the bottom to be mapped with problems, theories and concepts. This page provides us an overiview of a histography in a single. One can also interact with timeline to browse through events.

The backend database server has tables like author, concept, event, interpretation, period, problem, revolution and theory. These tables are backed by models in Backend layer. A seed script has data which populates default example of Cuban revolution. Each model is connected with intermediate tables to build connections between those models. All models have default scaffold built which can easily perform CRUD operations.

The application is deployed on a Heroku cloud server. It has out-of-box integration with Ruby On Rails and is easy to deploy with just one-click.

\chapter{Future Work}
Vistoria still needs a lot of work. Here is a list of things that should be added to the system to help improve user expierence:-
\section{Admin section}
For user authorization and data moderation.

\section{Data Entry Pages}
For end user to be able to create Vistoria on their own.

\section{Getting More information on Events}
Events can have more information that can be shown.

\section{Comments and Notes on Events}
Adding a comments and notes section will help users keep track histography.

\section{Author pages}
Author pages will general help potrary author's work.

\section{Multimedia Support}
Allow facility to upload multimedia like images, video, etc to an event or any other model.